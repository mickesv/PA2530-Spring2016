\documentclass[10pt]{beamer}

%% Use this for 4 on 1 handouts
%\documentclass[handout]{beamer}
%\usepackage{pgfpages}
%\pgfpagesuselayout{4 on 1}[landscape, a4paper, border shrink=5mm]

\usepackage[english]{babel}
\usepackage[utf8]{inputenc}
\usepackage[T1]{fontenc}
\def\subitem{\item[\hspace{1.5cm} -]}


% Set the presentation mode to BTH
\mode<presentation>
{
	\usetheme{BTH_msv}
	% Comment this if you do not want to reveal the bullets before they are going to be used
	\setbeamercovered{transparent}
}


% Information for the title page

\title[]{Background}
\subtitle{Cloud Computing}
% \date[]{}

\author[Mikael Svahnberg]{Mikael Svahnberg\inst{1}}
\institute[BTH] % (optional, but mostly needed)
{
  \inst{1}%
 Mikael.Svahnberg@bth.se\\
 School of Computing\\
 Blekinge Institute of Technology%
}

% Delete this, if you do not want the table of contents to pop up at
% the beginning of each subsection:
%\AtBeginSubsection[]
%{
%  \begin{frame}<beamer>{Outline}
%    \tableofcontents[currentsection,currentsubsection]
%  \end{frame}
%}


% If you wish to uncover everything in a step-wise fashion, uncomment
% the following command: 
%\beamerdefaultoverlayspecification{<+->}

\begin{document}

% Titlepage frame
\begin{frame}
  \titlepage
\end{frame}

% ToC frame
% Use \section and \subsection commands to get things into the ToC.
%\begin{frame}
 %\frametitle{Outline}
 % \tableofcontents
%\end{frame}

% -----------------------------
% Your frames goes here
% -----------------------------

    % - Cloud Background
    %    - History
    %    - Motivation
    %    - Principles
    %    - Legal Considerations


\begin{frame}[t]
\frametitle{Background}
Historically, as hardware requirements grew companies would:
\begin{itemize}
\item Host their own servers and pay for network access
\item Outsource to an IT service provider
\end{itemize}

Challenges with this:
\begin{itemize}
\item Expensive
\item Not Elastic (Contractually fixed service levels)
\item High entry barriers
\end{itemize}
\end{frame}


\begin{frame}[t]
\frametitle{Cloud Computing}
Characteristica that identifies a cloud service:
\begin{itemize}
\item Offered by a third party
\item Available when needed
\item Dynamically scalable
\item Low initial investment to get started
\item Pay for what you use, when you use it
\item Easily accessible
\end{itemize}
\end{frame}

\begin{frame}[t]
\frametitle{Five Main Principles}
\begin{itemize}
\item Pooled Computing Resources
\item Virtualised Computing Resources
\item Elastic Scaling up or down as needed
\item Automated creation/deletion of virtual machines
\item Resource usage billed only as used.
\end{itemize}

\ldots
\begin{itemize}
\item Note that \emph{most} of these are oriented towards the underlying technology.
\item Together, they enable a cloud provider to offer the service at a lower cost than for a company to host the servers themselves.
\item Ends up in the ability to offer the last bullet.
\item Network?
\item Access API
\item Storage + Databases
\end{itemize}
\end{frame}


\begin{frame}[t]
\frametitle{Cloud Benefits}

\begin{itemize}
\item Lower Initial barrier: Capital expenses $\rightarrow$ Operational expenses
\item Responsiveness: You do not have to wait for procurement of new servers.
\item Security: Dedicated staff whose sole business is to care about security issues
\subitem (Is this true for all types of cloud services?)
\item 
\end{itemize}
\end{frame}


\begin{frame}[t]
\frametitle{{\textbackslash}b[A-Z]A\{2\}S{\textbackslash}b}
\begin{itemize}
\item IAAS Infrastructure As A Service
\item PAAS Platform As A Service
\item SAAS Software As A Service
\item FAAS Framework As A Service
\item AAAS Application As A Service
\end{itemize}

Also:
\begin{itemize}
\item DAAS Private Clouds
\end{itemize}
\end{frame}

\begin{frame}[t]
\frametitle{Some popular Cloud Providers}
\begin{itemize}
\item Amazon EC2 (IAAS)
\item DigitalOcean (IAAS)
\item Microsoft Azure (IAAS)
\item Microsoft OneDrive (PAAS?)
\item Microsoft Office Online (AAAS)
\item Google Drive (PAAS?)
\item Google Docs (AAAS)
\item Google App Engine (PAAS)
\item Rackspace (IAAS)
\item Gmail/Yahoo/Outlook (AAAS)
\end{itemize}
\end{frame}


\begin{frame}[t]
\frametitle{Reasons for using the Cloud}
\begin{itemize}
\item Need more Capacity
\subitem Computing Power
\subitem Storage capacity
\subitem Burst-rate capacity
\subitem \ldots
\item Don't want to specialise on Server Maintenance
\subitem HW Repairs and Replacements
\subitem OS upgrades
\subitem Round-the-clock maintenance
\item Higher Reliability / Availability
\item Built-in and immediate Scalability (up as well as down)
\item Easier(?) Software Licensing
\end{itemize}
\end{frame}



\begin{frame}[t]
\frametitle{Challenges when Developing Cloud Applications}
\begin{itemize}[<+->]
\item Understand your needs: What service level do you require (How many servers, for how long? How often?)
\item Understanding your quality requirements. Why are you going to the cloud?
\subitem Hint: It is often more than one quality requirement that is important.
\item Designing a scalable software architecture
\item Setting up automated deployment of your application
\item Setting up equivalent and automated development / test / stage / deployment envonments.
\item Setting up automated Provisioning and Orchestration
\item Defining your database needs, selecting the right database, and design a (cloud-) scalable database design.
\item Security! Despite all the fancy promises, if you opt for IAAS or PAAS, you need to take care of this yourself!
\end{itemize}
\end{frame}

\begin{frame}[t]
\frametitle{Challenges when Developing Cloud Applications}

\begin{itemize}
\item Security!! You also need to protect your application from other applications running on the same cloud provider.
\item Security!!! Design your application to Protect/encrypt your data when it is on the cloud.
\end{itemize}

Also:
\begin{itemize}
\item What is the value of existing infrastructure? How does this influence your cost/value calculations?
\item Are your software licenses ``cloud friendly''?
\item Are you aware of the legal, regulatory, and standards that are relevant for your application
\end{itemize}
\end{frame}


\begin{frame}[t]
\frametitle{Characteristica of a Cloud Platform}

\begin{itemize}
\item 
\end{itemize}

% Cloud Architecture Patterns
% Enabled by (the illusion of ) infinite resources and limited by the maximum capacity of individual virtual machines, cloud scaling is horizontal.
% • Enabled by a short-term resource rental model, cloud scaling releases resources as easily as they are added.
% • Enabled by a metered pay-for-use model, cloud applications only pay for currently allocated resources and all usage costs are transparent.
% • Enabled by self-service, on-demand, programmatic provisioning and releasing of resources, cloud scaling is automatable.
% • Both enabled and constrained by multitenant services running on commodity hardware, cloud applications are optimized for cost rather than reliability; failure is routine, but downtime is rare.
% • Enabled by a rich ecosystem of managed platform services such as for virtual ma­ chines, data storage, messaging, and networking, cloud application development is simplified. 

% Leverages cloud-platform services for reliable, scalable infrastructure. (“Let the platform do the hard stuff.”)
% • Uses non-blocking asynchronous communication in a loosely coupled architecture.
% • Scales horizontally, adding resources as demand increases and releasing resources as demand decreases.
% • Cost-optimizes to run efficiently, not wasting resources.
% • Handles scaling events without downtime or user experience degradation.
% • Handles transient failures without user experience degradation.
% • Handles node failures without downtime.
% • Uses geographical distribution to minimize network latency.
% • Upgrades without downtime.
% • Scales automatically using proactive and reactive actions.
% • Monitors and manages application logs even as nodes come and go.

\end{frame}

% -----------------------------


%% All of the following is optional and typically not needed. 
%\appendix
%\begin{frame}[allowframebreaks]
%  \frametitle{For Further Reading}
%    
%  \begin{thebibliography}{10}
%    
%  \beamertemplatebookbibitems
%  % Start with overview books.

%  \bibitem{Author1990}
%    A.~Author.
%    \newblock {\em Handbook of Everything}.
%    \newblock Some Press, 1990.
%     
%  \beamertemplatearticlebibitems
%  % Followed by interesting articles. Keep the list short. 

%  \bibitem{Someone2000}
%    S.~Someone.
%    \newblock On this and that.
%    \newblock {\em Journal of This and That}, 2(1):50--100,
%    2000.
%  \end{thebibliography}
%\end{frame}

\end{document}



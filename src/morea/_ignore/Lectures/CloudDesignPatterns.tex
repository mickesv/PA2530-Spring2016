\documentclass[10pt]{beamer}

%% Use this for 4 on 1 handouts
%\documentclass[handout]{beamer}
%\usepackage{pgfpages}
%\pgfpagesuselayout{4 on 1}[landscape, a4paper, border shrink=5mm]

\usepackage[english]{babel}
\usepackage[utf8]{inputenc}
\usepackage[T1]{fontenc}
\def\subitem{\item[\hspace{1.5cm} -]}


% Set the presentation mode to BTH
\mode<presentation>
{
	\usetheme{BTH_msv}
	% Comment this if you do not want to reveal the bullets before they are going to be used
	\setbeamercovered{transparent}
}


% Information for the title page

\title[]{Cloud Design Patterns\footnote{Creatively borrowed from B. Wilder, \emph{``Cloud Architecture Patterns''}, O'Reilly, 2012.}}
\subtitle{}
% \date[]{}

\author[Mikael Svahnberg]{Mikael Svahnberg\inst{1}}
\institute[BTH] % (optional, but mostly needed)
{
  \inst{1}%
 Mikael.Svahnberg@bth.se\\
 School of Computing\\
 Blekinge Institute of Technology%
}

% Delete this, if you do not want the table of contents to pop up at
% the beginning of each subsection:
%\AtBeginSubsection[]
%{
%  \begin{frame}<beamer>{Outline}
%    \tableofcontents[currentsection,currentsubsection]
%  \end{frame}
%}


% If you wish to uncover everything in a step-wise fashion, uncomment
% the following command: 
%\beamerdefaultoverlayspecification{<+->}

\begin{document}

% Titlepage frame
\begin{frame}
  \titlepage
\end{frame}

% ToC frame
% Use \section and \subsection commands to get things into the ToC.
%\begin{frame}
 %\frametitle{Outline}
 % \tableofcontents
%\end{frame}

% -----------------------------
% Your frames goes here
% -----------------------------
\begin{frame}[t]
\frametitle{Scalability}

\begin{itemize}
\item (Flexible) Scalability is one of the core features of Cloud Computing
\item Vertical Scaling (increase capacity per node)
\item Horizontal Scaling (adding nodes)
\end{itemize}
\end{frame}


\begin{frame}[t]
\frametitle{Measures of Scalability}
A combination of:
\begin{itemize}
\item Concurrent Users
\item Response Time
\item Processed items / time unit
\item Complexity of processing requests
\end{itemize}
\end{frame}

\begin{frame}[t]
\frametitle{Issues that influence scalability}
Scarce resources:
\begin{itemize}
\item Computing power
\item RAM
\item Storage space
\item Network bandwidth
\end{itemize}
\end{frame}

\begin{frame}[t]
\frametitle{Scaling Mindsets}
\begin{itemize}
\item Cetain mindsets help in addressing Cloud Scaling 
\item These do not affect the architecture \emph{per se}, but influences your choices of solutions.
\begin{itemize}
\item Eventual Consistency
\item Multitenancy
\item Inevitable Failure
\item Network Latency
\end{itemize}
\end{itemize}
\end{frame}

\begin{frame}[t]
\frametitle{Eventual Consistency}
\begin{itemize}[<+->]
\item \emph{``At any moment, most of an eventially consistent database is consistent, with somne small number of values still being updated''}
\item CAP theorem: \{Consistency, Availability, Partition Tolerance\}: \emph{Pick two!}
\item This implies:
\subitem Data is always(?) available, although not always 100\% correct
\subitem Your system needs to robustely deal with this.
\item Compare with RDBMS' \emph{ACID} property.
\item With a distributed database (using e.g. a NoSQL database), you instead have \emph{BASE}:
\begin{itemize}
\item Basically Available
\item Soft State
\item Eventually Consistent
\end{itemize}
\end{itemize}
\end{frame}

\begin{frame}[t]
\frametitle{Multitenancy}
\begin{itemize}
\item One company (host) operates the application for use by other companies (tenants)
\item The tenants have the impression that they are alone in using the service
\item Has implications on:
\subitem data partitioning
\subitem security
\subitem performance management
\item As much a concern for the cloud provider as for the cloud application provider.
\end{itemize}
\end{frame}

\begin{frame}[t]
\frametitle{Inevitable Failure}
\begin{itemize}
\item The cloud provider is likely to use cheap \emph{commodity hardware}
\item Therefore, hardware failure is inevitable (although not necessarily frequent)
\item This implies that your application need to be \emph{robust}
\item Focus shift from MTBF to MTTR
\end{itemize}
\end{frame}

\begin{frame}[t]
\frametitle{Network Latency}
\begin{itemize}
\item Problem: You are \emph{here}, your data are \emph{there}, and your users are \emph{yonder}
\item \ldots And the servers running your applications are neither \emph{here nor there}.
\item Moving data between your servers and, eventually, to the users requires network bandwidth
\item Strategies your application may take:
\begin{itemize}
\item Data compression
\item Background processing
\item Predictive fetching
\item Moving your application closer to the users
\item Moving the data closer to the users
\item Moving data processing nodes closer together
\end{itemize}
\end{itemize}
\end{frame}


\begin{frame}[t]
\frametitle{Scalability Patterns}
\begin{itemize}
\item Generic Scalability:
\begin{itemize}
\item Horizontally Scaling Compute Pattern
\item Queue-Centric Workflow Pattern
\item Auto-Scaling Pattern
\end{itemize}
\item Eventual Consistency
\begin{itemize}
\item MapReduce Pattern
\item Database Sharding Pattern
\end{itemize}
\item Multitenancy and Inevitable Failure
\begin{itemize}
\item Busy Signal Pattern
\item Node Failure Pattern
\end{itemize}
\item Network Latency
\begin{itemize}
\item Colocate Pattern
\item Valet Key Pattern
\item CDN Pattern
\item Multisite Deployment Pattern
\end{itemize}
\end{itemize}
\end{frame}


\begin{frame}[t]
\frametitle{Challenges with Horizontal Scaling}
\begin{itemize}
\item Load Balancing
\item Synchronisation between nodes
\item Managing Session State
\subitem Sticky sessions? (What happens if that node breaks?)
\subitem Cookies (for small amounts of data)
\subitem Cookies (as a key to the full db record)
\item Capacity Planning (per time unit)
\item Sizing the virtual machines
\end{itemize}
\end{frame}


\begin{frame}[t]
\frametitle{Queue Centric Workflow}
\begin{itemize}
\item Problem: Some jobs take longer time. This may impact the responsiveness of the application.
\item Solution:
\subitem Package the tasks to do in a job description and add it to a queue.
\subitem Worker(s) in the service tier picks work from the job queue and processes them in due order.
\subitem cf. Eventual Consistency
\item Challenges:
\subitem One worker picks a job but fails halfway through.
\subitem Solutions: Invisibility window, idempotent processing (for repeat messages), handling of poison messages
\end{itemize}
\end{frame}


% % trivial?
% \begin{frame}[t]
% \frametitle{Auto-Scaling Pattern}
% \begin{itemize}
% \item Problem: Automate the scaling so that you do not have to manually spin up new servers to cope with (scheduled) fluctuations in demand.

% \end{itemize}
% \end{frame}

\begin{frame}[t]
\frametitle{MapReduce}
\begin{itemize}
\item Map: execute function on each instance of the data
\item Reduce: merge the results of a map into a combined and consistent data set again.
\item Challenges:
\subitem moving big data takes time and is expensive.
\subitem Solution: ``bring the compute to the data''
\end{itemize}
\end{frame}

\begin{frame}[t]
\frametitle{Database Sharding}
\begin{itemize}
\item Classic database division: Vertical
\item For example: db\{users\}, db\{orders\}, db\{warehouse\}, \ldots
\item Sharding divides the data horizontally.
\item All db instances have the entire schema, and contains a subset of all the rows.
\item One db row only exist in one db.
\item Challenges:
\subitem Deciding how to shard your data to be most efficient
\subitem Cloudfronting (where should a particular row be?)
\subitem Defining the shards to minimise database queries over several shards, or shards far away.
\end{itemize}
\end{frame}

\begin{frame}[t]
\frametitle{Node Failure}
\begin{tabular}{lll}
Scenario & Warning & Impact\\
\hline
Sudden failure & no & local data is lost\\
Platform shutdown/restart & yes & local data may be available\\
Application shutdown/restart & yes & local data is available\\
Shutdown/destroy & yes & local data is lost\\
\hline
\end{tabular}

\vspace{0.5cm}
Advice:
\begin{itemize}
\item Treat all interruptions as node failures
\item Maintain sufficient capacity for failure \emph{(N+1 rule)}
\item Load balancing to minimise interruption for user
\item Combine with other patterns, e.g. queue-centric workflow pattern, busy signal pattern
\end{itemize}
\end{frame}

\begin{frame}[t]
\frametitle{Summary}
\begin{itemize}
\item There are a few mindsets that influence your choices of solutions for a cloud application
\begin{scriptsize}
\begin{itemize}
\item Eventual Consistency
\item Multitenancy
\item Inevitable Failure
\item Network Latency
\end{itemize}
\end{scriptsize}
\item Related to these mindsets, there are a number of design patterns for addressing them in a cloud setting.
\item Some of these are mentioned in this lecture.
\item Some of these are discussed in some further detail.
\item As usual, when discussing architecture and design patterns, the details are not available until you design \emph{your specific application}.
\end{itemize}
\end{frame}


% -----------------------------


%% All of the following is optional and typically not needed. 
%\appendix
%\begin{frame}[allowframebreaks]
%  \frametitle{For Further Reading}
%    
%  \begin{thebibliography}{10}
%    
%  \beamertemplatebookbibitems
%  % Start with overview books.

%  \bibitem{Author1990}
%    A.~Author.
%    \newblock {\em Handbook of Everything}.
%    \newblock Some Press, 1990.
%     
%  \beamertemplatearticlebibitems
%  % Followed by interesting articles. Keep the list short. 

%  \bibitem{Someone2000}
%    S.~Someone.
%    \newblock On this and that.
%    \newblock {\em Journal of This and That}, 2(1):50--100,
%    2000.
%  \end{thebibliography}
%\end{frame}

\end{document}



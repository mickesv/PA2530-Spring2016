\documentclass[10pt]{beamer}

%% Use this for 4 on 1 handouts
%\documentclass[handout]{beamer}
%\usepackage{pgfpages}
%\pgfpagesuselayout{4 on 1}[landscape, a4paper, border shrink=5mm]

\usepackage[english]{babel}
\usepackage[utf8]{inputenc}
\usepackage[T1]{fontenc}
\usepackage{marvosym}
\DeclareUnicodeCharacter{20AC}{\EUR{}}

\def\subitem{\item[\hspace{1.5cm} -]}


% Set the presentation mode to BTH
\mode<presentation>
{
	\usetheme{BTH_msv}
	% Comment this if you do not want to reveal the bullets before they are going to be used
	\setbeamercovered{transparent}
}


% Information for the title page

\title[]{The Cloud Business Case}
\subtitle{}
% \date[]{}

\author[Mikael Svahnberg]{Mikael Svahnberg\inst{1}}
\institute[BTH] % (optional, but mostly needed)
{
  \inst{1}%
 Mikael.Svahnberg@bth.se\\
 School of Computing\\
 Blekinge Institute of Technology%
}

% Delete this, if you do not want the table of contents to pop up at
% the beginning of each subsection:
%\AtBeginSubsection[]
%{
%  \begin{frame}<beamer>{Outline}
%    \tableofcontents[currentsection,currentsubsection]
%  \end{frame}
%}


% If you wish to uncover everything in a step-wise fashion, uncomment
% the following command: 
%\beamerdefaultoverlayspecification{<+->}

\begin{document}

% Titlepage frame
\begin{frame}
  \titlepage
\end{frame}

% ToC frame
% Use \section and \subsection commands to get things into the ToC.
%\begin{frame}
 %\frametitle{Outline}
 % \tableofcontents
%\end{frame}

% -----------------------------
% Your frames goes here
% -----------------------------

\begin{frame}[t]
\frametitle{Economics}

\begin{itemize}
\item I know. Boring, but it must be said.
\item This is what you need to do in order to argue your case for your boss.
\end{itemize}

A Scale of Different Deployment Models:
\begin{itemize}
\item Traditional Internal IT
\only<2>{\subitem All IT infrastructure is is capital expenditure}
\item Colocation
\only<3>{
\subitem You pay for the hardware,
\subitem but place it at a colocation facility
\subitem Facility provides Power, Cooling, Rack Space, Network connectivity, Backup power, Physical Securuty
\subitem Turns some of these into operational expenditures
}
\item Managed Service
\only<4>{
\subitem As Colocation, but you also rent the servers and networking hardware
}
\item Cloud Model
\only<5>{
\subitem As in a Managed Service, but you rent \emph{virtualised resources}.
\subitem Therefore, you only pay for what you use.
}
\end{itemize}

\end{frame}

\begin{frame}[t]
\frametitle{Example}
Let's say you need:
\begin{itemize}
\item 2 firewalls: 2*1500€ = 3000€
\item 6 commodity servers: 6*3000€ = 18 000€
\end{itemize}

You also need (not counted in example):
\begin{itemize}
\item a Room to keep your stuff in
\item an Internet Connection
\item a Rack Cabinet
\item a Network Switch
\item Load Balancing
\item Cooling
\item Someone managing the hardware
\item Licenses for your software
\item \ldots
\end{itemize}

\end{frame}

\begin{frame}[t]
\frametitle{Example: Internal Deployment}
\begin{tabular}{rrl}
&& Internal Deployment\\
\hline
  &  3 000€ & Firewalls\\
+ & 18 000€ & Servers\\
\cline{2-2}\\
= & 21 000€ & Total Capital Expenditure\\
\cline{2-2}\\
/ &      36 & Depreciation over 3 years\\
\cline{2-3}\\
= &    600€ & Cost per month\\
\hline
\end{tabular}
\end{frame}

\begin{frame}[t]
\frametitle{Example: Cloud Deployment}
\begin{tabular}{rrl}
&& Cloud Deployment\\
\hline
  &  20\$ & per month for 2 Firewalls / Load Balancers\\
+ & 60\$ & per month 6 Servers\\
\cline{2-2}\\
= & 80\$ & Total Operational Expenditure \emph{per month}\\
\hline
\end{tabular}
\end{frame}


\begin{frame}[t]
\frametitle{Understand your requirements}
\begin{itemize}
\item In order to make your business case, you need to \emph{understand your requirements}
\item Understanding your requirements is about understanding the \emph{quality requirements} of your cloud application.
\item Some quality requirements are more in focus than others 
\end{itemize}
\end{frame}

\begin{frame}[t]
\frametitle{Cloud Quality Requirements}
\begin{itemize}
\item Scalability
\item Reliability / Availability
\item Performance
\subitem Storage
\subitem Capacity
\subitem Bandwidth
\item Security
\item Privacy
\item Cost Optimisation
\end{itemize}
\end{frame}

\begin{frame}[t]
\frametitle{Quality Requirements are Time Dependent}
\begin{itemize}
\item In traditional deployment, you pick one service level and dimension your solution accordingly.
\item This may mean a trade-off that during extreme peak hours your consciously do not meet your quality requirements targets.
\item In a cloud solution, you can always spin up more machines as a consequence of (planned) peaks.
\item Therefore, you need to view your quality requirements over time -- often with a per-hour granularity.
\end{itemize}
\end{frame}

\begin{frame}[t]
\frametitle{Example, Scalability}
\centering
\only<1>{
\includegraphics[width=10cm]{FusersMonth.pdf}\\
{\scriptsize Average number of Concurrent Users per Month, expected growth scenario.}
}

\only<2>{
\includegraphics[width=10cm]{FusersDay.pdf}\\
{\scriptsize Number of Concurrent Users per Hour and Day, measured.}
}

\end{frame}

\begin{frame}[t]
\frametitle{Scalability}
\begin{itemize}
\item Max / median number of concurrent users
\item Max / median acceptable response time
\item Burst rates \& times
\item Max / median latency
\end{itemize}

At any given time, this influences:
\begin{itemize}
\item The number of required servers
\item The computing power of these
\item The bandwidth requirements
\item The speed of your storage devices
\item The size of your storage devices
\end{itemize}

\end{frame}

\begin{frame}[t]
\frametitle{Reliability and Availability}
\begin{itemize}
\item Transient failures
\item Upgrades without downtime
\item Continuous monitoring and logging of application’s health
\item Backups
\item Recovery
\item Migration
\item Data persistency
\end{itemize}

\only<1>{
Availability (may) require:
\begin{itemize}
\item That your cloud resources have sufficient storage for backups
\item That you have additional storage resources for long-term backups
\item That you use (slow) long-term storage in tandem with faster storage solutions.
\item That you distribute your application over several datacenters
\item That you implement loadbalancing between your servers on different datacenters.
\end{itemize}
}
\only<2>{
Also:
\begin{itemize}
\item What does the cloud provider promise in terms of uptime?
\end{itemize}
}
\end{frame}

\begin{frame}[t]
\frametitle{Performance}
\begin{itemize}
\item Similar to Scalability:
\item Computing power
\item Storage response times
\item Storage capacity
\item Network bandwidth
\end{itemize}

At any given time, this influences:
\begin{itemize}
\item The number of required servers
\item The computing power of these
\item The bandwidth requirements
\item The speed of your storage devices
\item The size of your storage devices
\end{itemize}

\end{frame}

\begin{frame}[t]
\frametitle{Security}
\begin{itemize}
\item Data security
\item Hosts security
\item Network security
\end{itemize}

Things to look out for:
\begin{itemize}
\item What promises do your cloud provider make wrt. storage persistency?
\item What services are provided to maintain your server platform, especially security patches?
\item Is it possible for other users of the same cloud vendor to get at your site ``from behind''?
\end{itemize}

\end{frame}

\begin{frame}[t]
\frametitle{Privacy}
\begin{itemize}
\item Where is your data stored?
\item Under what circumstances would your cloud provider have to give up your data (e.g. as a response to a subpoena)?
\item What measures do *you* need to take to protect the privacy of your users?
\item What do your cloud provider promise?
\end{itemize}
\end{frame}

\begin{frame}[t]
\frametitle{Summary}
\begin{itemize}
\item Your quality requirements determine:
\begin{itemize}
\item the cloud infrastructure you need
\item the support structure around the servers you need from your cloud provider
\end{itemize}
\item In turn, this determines what you must pay
\item Arguing for \emph{why} you need that service level, and what it will cost you is your business case.
\item \emph{Best(?) Alternative Investment}: What will it cost you to host your servers yourself?
\item Remember: With a cloud solution you can have a much finer time granularity, and spin up servers to only deal with e.g. peak hours.
\end{itemize}
\end{frame}




    % Number of required machines
    % Computing power of machines
    % Storage requirements
    % Bandwidth requirements
    % Domain name(s)
    % Specific security requirements
    % Specific privacy requirements


    % Scalability, in particular being flexible in terms of:
    %     Max / median number of concurrent users
    %     Max / median acceptable response time
    %     Burst rates & times
    %     Max / median latency
    % Reliability / Availability, in particular your requirements on being able to handle:
    %     Transient failures
    %     Upgrades without downtime
    %     Continuous monitoring and logging of application’s health
    %     Backups
    %     Recovery
    %     Migration
    %     Data persistency
    % Performance, your base and growth requirements on:
    %     Computing power
    %     Storage
    % Security, your particular requirents for security of:
    %     Data
    %     Your hosts
    %     Your network (or, to be specific, your cloud provider’s network).
    % Privacy
    % Cost Optimisation




% -----------------------------


%% All of the following is optional and typically not needed. 
%\appendix
%\begin{frame}[allowframebreaks]
%  \frametitle{For Further Reading}
%    
%  \begin{thebibliography}{10}
%    
%  \beamertemplatebookbibitems
%  % Start with overview books.

%  \bibitem{Author1990}
%    A.~Author.
%    \newblock {\em Handbook of Everything}.
%    \newblock Some Press, 1990.
%     
%  \beamertemplatearticlebibitems
%  % Followed by interesting articles. Keep the list short. 

%  \bibitem{Someone2000}
%    S.~Someone.
%    \newblock On this and that.
%    \newblock {\em Journal of This and That}, 2(1):50--100,
%    2000.
%  \end{thebibliography}
%\end{frame}

\end{document}


